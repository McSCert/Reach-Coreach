% !TeX encoding = UTF-8
% !TeX spellcheck = en_GB
\documentclass[12pt,letterpaper]{report}
\usepackage[utf8]{inputenc}
%\usepackage[toc,page]{appendix}
\usepackage{amsmath}
\usepackage{graphicx}
\usepackage[destlabel,pdfusetitle]{hyperref}
\usepackage{hhline} % for tabular expressions
\usepackage{multirow} % "
\usepackage{algorithmic} % code in the tabular expression explanation
\usepackage{caption}
\usepackage{subcaption} % subfigures
\usepackage{geometry} % to adjust margins
\usepackage{longtable}
\usepackage{tabu}
\usepackage{matlab-prettifier}
\lstset{style=Matlab-editor}
\usepackage{comment}
\usepackage{appendix}
\usepackage[T1]{fontenc}
\usepackage{bigfoot} % to allow verbatim in footnote
\usepackage{float}
\usepackage{xspace} % http://ctan.org/pkg/xspace (for macros)

\renewcommand{\contentsname}{Table of Contents}

\newcommand{\Appendixautorefname}{Appendix}%
\renewcommand{\chapterautorefname}{Chapter}%
\renewcommand{\sectionautorefname}{Section}%
\renewcommand{\subsectionautorefname}{Section}%
\renewcommand{\subsubsectionautorefname}{Section}%
\setcounter{secnumdepth}{3}

\makeatletter
\newcommand{\mylabel}[1]{\raisebox{\f@size pt}{\phantomsection}\label{#1}}
\makeatother

\newcommand{\PushDownLink}{The Data Store Push-Down Tool}
%Requirement links
\newcommand{\TestingInputsOutputsLink}{\hyperref{srs.pdf}{}{subsec:TestingInputsOutputs}{the Testing Inputs \& Outputs Requirement}}
\newcommand{\IntegrationLink}{\hyperref{srs.pdf}{}{subsec:Integration}{the Integration Requirement}}
\newcommand{\SOCReqLink}{\hyperref{srs.pdf}{}{subsec:SOC}{the SOC Requirement}}
\newcommand{\BPCMFlowchartLink}{\hyperref{srs.pdf}{}{fig:BPCMflowchart}{the BPCM Flowchart}}
\hypersetup{
	linktocpage,
	colorlinks,
	linkcolor=blue,
	urlcolor=blue,
	pdfkeywords={{Reach/Coreach} {documentation} {Simulink} {McMaster University}},
	pdfstartpage=2
}

\begin{document}
	
	\title{Reach.Coreach Tool: Software Design Description (Draft)}
	
	\author{
		Gordon Marks\\
	}
	
	\maketitle
	
	\tableofcontents
	
	\chapter{Introduction}
	
	\par This document is a Software Design Description (SDD) for the Reach/Coreach Tool. 
	The purpose of the tool is to perform an impact analysis from a selection of blocks and/or signal lines with the model containing those blocks.
	The tool is also used to perform model slices to reduce the model to the relevant blocks and signals. 
%	A high level hierarchy design is presented which groups similar functions and scripts based on their functionality. 
	
%	\par The outline of the document is as follows. 
%	Chapter~\ref{TopLevelDesign} presents an architectural view of the tool. 
%	Chapter~\ref{Modules} presents the detailed design of the tool analysis component of the tool. 
%	Chapter~\ref{DetailedDesign} presents an overview of the functions and what they do via the comments at the top of each function.
	
	\chapter{Top Level Design Overview} \label{TopLevelDesign}
	
	\par The ReachCoreach Tool is primarily focussed in the main ReachCoreach class file contains the bulk of the analyses, but uses a number of functions within to complete a variety of functionalities.
	The ``find'' functions (the files named starting with ``find'' and ending with ``RCR'') are used to identify implicit connections between Gotos, Froms, Tag Visibilities, Data Store Reads, Writes, and Memories.
%	Therefore, the functions that have a similar functionality, or functions that have similar applications, are grouped together as modules as shown below:
	
%	\begin{figure}[h]
%		\centering
%		\includegraphics[width=1\linewidth]{figs/<<>>.pdf}
%		\caption{Reach/Coreach Tool Overview}
%		\label{fig:chart}
%	\end{figure}
%	
%	\par Each block represents a module and the header of the block (blue colour) indicates the name of the module. The blocks also contain the name of the functions which belong to that module. Furthermore, a uses relationship between modules is represented by arrows. This describes the general architecture of the functions used in the Boundary Diagram Tool.
	
%	\chapter{Module Break-down} \label{Modules}
%	
%	
%	\begin{itemize}
%		
%	\end{itemize}
	
	\section{Detailed Design}
	
	\subsection{sl\_customization}
% >> Start auto gen <<
\begin{lstlisting}
function sl_customization(cm)
\end{lstlisting}
% >> End auto gen <<
	The sl\_customization file is responsible for handling the options available from the Simulink context menu (when you right-click) under Reach/Coreach.
	
	\subsection{ReachCoreach}
% >> Start auto gen <<
\begin{lstlisting}
classdef ReachCoreach < handle
    % REACHCOREACH A class that enables performing reachability/coreachability
    % analysis on blocks in a model.
    %
    %   A reachability analysis (reach) finds all blocks and lines that the
    %   given initial blocks affect via control flow or data flow. A
    %   coreachability analysis (coreach) finds all blocks and lines that affect
    %   the given initial blocks via control flow or data flow. After creating a
    %   ReachCoreach object, the reachAll and coreachAll methods can be used to
    %   perform these analyses respectively, and highlight all the
    %   blocks/lines in the reach and coreach.
    %
    % Example:
    %       open_system('ReachCoreachDemo_2011')
    %       r = ReachCoreach('ReachCoreachDemo_2011');
    %
    %   % Perform a reachability analysis:
    %       r.reachAll({'ReachCoreachDemo_2011/In2'},[]);
    %
    %   % Clear highlighting:
    %       r.clear();
    %
    %   % Change the highlighting colors
    %       r.setColor('blue', 'magenta')
    %
    %   % Perform a coreachability analysis:
    %       r.coreachAll({'ReachCoreachDemo_2011/Out2', {'ReachCoreachDemo_2011/Out3'},{});
    %
    %   % Perform a slice:
    %       r.slice();
\end{lstlisting}
% >> End auto gen <<
	The ReachCoreach file defines the ReachCoreach class, whenever the tool is used, an object of this class will be created.
	The following nested sections refer to functions defined in the ReachCoreach file.
		
		\subsubsection{getColor}
% >> Start auto gen <<
\begin{lstlisting}
        function [fgcolor, bgcolor] = getColor(object)
            % GETCOLOR Get the highlight colours for the reach/coreach.
            %
            %   Inputs:
            %       object  ReachCoreach object.
            %
            %   Outputs:
            %       fgcolor Foreground colour.
            %       bgcolor Background colour.
            %
            %   Example:
            %       obj.getColor()
\end{lstlisting}
% >> End auto gen <<
		
		\subsubsection{setColor}
% >> Start auto gen <<
\begin{lstlisting}
        function setColor(object, color1, color2)
            % SETCOLOR Set the highlight colours for the reach/coreach.
            %
            %   Inputs:
            %       object  ReachCoreach object.
            %       color1  Parameter for the highlight foreground colour.
            %               Accepted values are 'red', 'green', 'blue', 'cyan',
            %               'magenta', 'yellow', 'black', 'white'.
            %
            %       color2  Parameter for the highlight background colour.
            %               Accepted values are 'red', 'green', 'blue', 'cyan',
            %               'magenta', 'yellow', 'black', 'white'.
            %
            %   Outputs:
            %       N/A
            %
            %   Example:
            %       obj.setColor('red', 'blue')
\end{lstlisting}
% >> End auto gen <<
		
		\subsubsection{setHiliteFlag}
% >> Start auto gen <<
\begin{lstlisting}
        function setHiliteFlag(object, flag)
            % SETHILITEFLAG Set hiliteFlag object property. Determines whether
            % to hilite objects or not.
            %
            %   Inputs:
            %       object  ReachCoreach object.
            %       flag    Boolean value to set the flag to.
            %
            %   Outputs:
            %       N/A
            %
            %   Example:
            %       obj.setHiliteFlag(true)
\end{lstlisting}
% >> End auto gen <<
		
		\subsubsection{hiliteObjects}
% >> Start auto gen <<
\begin{lstlisting}
        function hiliteObjects(object)
            % HILITEOBJECTS Highlight the reached/coreached blocks and lines.
            %
            %   Inputs:
            %       object  ReachCoreach object.
            %
            %   Outputs:
            %       N/A
            %
            %   Example:
            %       obj.hiliteObjects()
\end{lstlisting}
% >> End auto gen <<
		
		\subsubsection{slice}
% >> Start auto gen <<
\begin{lstlisting}
        function slice(object)
            % SLICE Isolate the reached/coreached blocks by removing
            % unhighlighted blocks.
            %
            %   Inputs:
            %       object  ReachCoreach object.
            %
            %   Outputs:
            %       N/A
            %
            %   Example:
            %       obj.slice()
\end{lstlisting}
% >> End auto gen <<
		
		\subsubsection{clear}
% >> Start auto gen <<
\begin{lstlisting}
        function clear(object)
            % CLEAR Remove all reach/coreach highlighting.
            %
            %   Inputs:
            %       object  ReachCoreach object.
            %
            %   Outputs:
            %       N/A
            %
            %   Example:
            %       obj.clear()
\end{lstlisting}
% >> End auto gen <<
		
		\subsubsection{reachAll}
% >> Start auto gen <<
\begin{lstlisting}
        function reachAll(object, selection, selLines)
            % REACHALL Reach from a selection of blocks.
            %
            %   Inputs:
            %       object      ReachCoreach object.
            %       selection   Cell array of strings representing the full
            %                   names of blocks.
            %       selLines    Array of line handles.
            %
            %   Outputs:
            %       N/A
            %
            %   Example:
            %       obj.reachAll({'ModelName/In1', 'ModelName/SubSystem/Out2'}, [])
\end{lstlisting}
% >> End auto gen <<
		
		\subsubsection{coreachAll}
% >> Start auto gen <<
\begin{lstlisting}
        function coreachAll(object, selection, selLines)
            % COREACHALL Coreach from a selection of blocks.
            %
            %   Inputs:
            %       object      ReachCoreach object.
            %       selection   Cell array of strings representing the full
            %                   names of blocks.
            %       selLines    Array of line handles.
            %
            %   Outputs:
            %       N/A
            %
            %   Example:
            %       obj.coreachAll({'ModelName/In1', 'ModelName/SubSystem/Out2'})
\end{lstlisting}
% >> End auto gen <<
		
		\subsubsection{reach}
% >> Start auto gen <<
\begin{lstlisting}
        function reach(object, port)
            % REACH Find the next ports to call the reach from, and add all
            % objects encountered to Reached Objects.
            %
            %   Inputs:
            %       object  ReachCoreach object.
            %       port
            %
            %   Output:
            %       N/A
\end{lstlisting}
% >> End auto gen <<
		
		\subsubsection{coreach}
% >> Start auto gen <<
\begin{lstlisting}
        function coreach(object, port)
            % COREACH Find the next ports to find the coreach from, and add all
            % objects encountered to coreached objects.
            %
            %   Inputs:
            %       object  ReachCoreach object.
            %       port
            %
            %   Outputs:
            %       N/A
\end{lstlisting}
% >> End auto gen <<
		
		\subsubsection{findIterators}
% >> Start auto gen <<
\begin{lstlisting}
        function iterators = findIterators(object)
            % FINDITERATORS Find all while and for iterators that need to be
            % coreached.
            %
            %   Inputs:
            %       object  ReachCoreach object.
            %       port
            %
            %   Outputs:
            %       N/A
\end{lstlisting}
% >> End auto gen <<
		
		\subsubsection{findSpecialPorts}
% >> Start auto gen <<
\begin{lstlisting}
        function findSpecialPorts(object)
            % FINDSPECIALPORTS Find all actionport, foreach, triggerport, and
            % enableport blocks and adds them to the coreach, as well as adding
            % their corresponding port in the parent subsystem block to the list
            % of ports to traverse.
            %
            %   Input:
            %       object  ReachCoreach object.
            %
            %   Outputs:
            %       N/A
\end{lstlisting}
% >> End auto gen <<
		
		\subsubsection{reachEverythingInSub}
% >> Start auto gen <<
\begin{lstlisting}
        function reachEverythingInSub(object, system)
            % REACHEVERYTHINGINSUB Add all blocks and outports of blocks in the
            % subsystem to the lists of reached objects. Also find all interface
            % going outward (outports, gotos, froms) and find the next
            % blocks/ports as if being reached by the main reach function.
            %
            %   Inputs:
            %       object ReachCoreach object.
            %       system
            %
            %   Outputs:
            %       N/A
\end{lstlisting}
% >> End auto gen <<
		
		\subsubsection{getInterfaceIn}
% >> Start auto gen <<
\begin{lstlisting}
        function blocks = getInterfaceIn(object, subsystem)
            % GETINTERFACEIN Get all the source blocks for the subsystem,
            % including Gotos and Data Store Writes.
            %
            %   Inputs:
            %       object      ReachCoreach object.
            %       subsystem
            %
            %   Outputs:
            %       blocks
\end{lstlisting}
% >> End auto gen <<
		
		\subsubsection{getInterfaceOut}
% >> Start auto gen <<
\begin{lstlisting}
        function blocks = getInterfaceOut(object, subsystem)
            % GETINTERFACEOUT Get all the destination blocks for the subsystem,
            % including Froms and Data Store Reads.
            %
            %   Inputs:
            %       object      ReachCoreach object.
            %       subsystem
            %
            %   Output:
            %       blocks
\end{lstlisting}
% >> End auto gen <<
		
		\subsubsection{traverseBusForwards}
% >> Start auto gen <<
\begin{lstlisting}
        function [path, exit] = traverseBusForwards(object, creator, oport, signal, path)
            % TRAVERSEBUSFORWARDS Go until a Bus Creator is encountered. Then,
            % return the path taken there as well as the exiting port.
            %
            %   Inputs:
            %       object  ReachCoreach object.
            %       creator
            %       oport
            %       signal
            %       path
            %
            %   Outputs:
            %       path
            %       exit
\end{lstlisting}
% >> End auto gen <<
		
		\subsubsection{traverseBusBackwards}
% >> Start auto gen <<
\begin{lstlisting}
        function [path, blockList, exit] = traverseBusBackwards(object, iport, signal, path, blockList)
            % TRAVERSEBUSBACKWARDS Go until Bus Creator is encountered. Then,
            % return the path taken there as well as the exiting port.
            %
            %   Inputs:
            %       object      ReachCoreach object.
            %       iport
            %       signal
            %       path
            %       blockList
            %
            %   Outputs:
            %       path
            %       blockList
            %       exit
\end{lstlisting}
% >> End auto gen <<
		
		\subsubsection{addToMappedArray}
% >> Start auto gen <<
\begin{lstlisting}
        function addToMappedArray(object, property, key, handle)
            % ADDTOMAPPEDARRAY
            %
            %   Inputs:
            %       object      ReachCoreach object.
            %       property
            %       key
            %       handle
            %
            %   Outputs:
            %
\end{lstlisting}
% >> End auto gen <<
		
	\subsection{findFromsInScopeRCR}
% >> Start auto gen <<
\begin{lstlisting}
function froms = findFromsInScopeRCR(obj, block, flag)
% FINDFROMSINSCOPE Find all the From blocks associated with a Goto block.
%
% 	Inputs:
% 		obj    The reachcoreach object containing goto tag mappings
%       block  The goto block of interest
%       flag   The flag indicating whether shadowing visibility tags are in the
%              model
%
% 	Outputs:
%		froms    The tag visibility block corresponding to input "block"
\end{lstlisting}
% >> End auto gen <<
	\subsection{findGotosInScopeRCR}
% >> Start auto gen <<
\begin{lstlisting}
function goto = findGotosInScopeRCR(obj, block, flag)
% FINDGOTOSINSCOPE Find the Goto block associated with a From block.
%
% 	Inputs:
% 		obj    The reachcoreach object containing goto tag mappings
%       block  The from block of interest
%       flag   The flag indicating whether shadowing visibility tags are in the
%              model
%
% 	Outputs:
%		goto   The goto block corresponding to input "block"
\end{lstlisting}
% >> End auto gen <<
	\subsection{findGotoFromsInScopeRCR}
% >> Start auto gen <<
\begin{lstlisting}
function blockList = findGotoFromsInScopeRCR(obj, block, flag)
% FINDGOTOFROMSINSCOPE Find all the Goto and From blocks associated with a 
% Goto Tag Visibility block.
%
% 	Inputs:
% 		obj        The reachcoreach object containing goto tag mappings
%       block      The tag visibility block of interest
%       flag       The flag indicating whether shadowing visibility tags 
%                  are in the model
%
% 	Outputs:
%		blockList  Cell array of goto/from blocks corresponding to input "block"
\end{lstlisting}
% >> End auto gen <<
	\subsection{findVisibilityTagRCR}
% >> Start auto gen <<
\begin{lstlisting}
function visBlock = findVisibilityTagRCR(obj, block, flag)
% FINDVISIBILITYTAG Find the Goto Visibility Tag block associated with a
% scoped Goto or From block.
%
% 	Inputs:
% 		obj       The reachcoreach object containing goto tag mappings
%       block     The goto or from block of interest
%       flag      The flag indicating whether shadowing goto tags are in the
%                 model
%
% 	Outputs:
%		visBlock  The tag visibility block corresponding to input "block"
\end{lstlisting}
% >> End auto gen <<
	\subsection{findReadsInScopeRCR}
% >> Start auto gen <<
\begin{lstlisting}
function reads = findReadsInScopeRCR(obj, block, flag)
% FINDREADSINSCOPE Find all the Data Store Read blocks associated with a Data
% Store Write block.
%
% 	Inputs:
% 		obj    The reachcoreach object containing data store mappings
%       block  The write block of interest
%       flag   The flag indicating whether shadowing data stores are in the
%              model
%
% 	Outputs:
%		froms    Thedata store read corresponding to input "block"
\end{lstlisting}
% >> End auto gen <<
	\subsection{findWritesInScopeRCR}
% >> Start auto gen <<
\begin{lstlisting}
function writes = findWritesInScopeRCR(obj, block, flag)
% FINDWRITESINSCOPE Find all the Data Store Writes associated with a Data
% Store Read block.
%
% 	Inputs:
% 		obj    The reachcoreach object containing data store mappings
%       block  The read block of interest
%       flag   The flag indicating whether shadowing data stores are in the
%              model
%
% 	Outputs:
%		froms    The data store write corresponding to input "block"
\end{lstlisting}
% >> End auto gen <<
	\subsection{findReadWritesInScopeRCR}
% >> Start auto gen <<
\begin{lstlisting}
function blockList = findReadWritesInScopeRCR(obj, block, flag)
% FINDREADWRITESINSCOPE Find all the Data Store Read and Data Store Write 
% blocks associated with a Data Store Memory block.
%
% 	Inputs:
% 		obj        The reachcoreach object containing data store mappings
%       block      The data store memory block of interest
%       flag       The flag indicating whether shadowing data stores are in the
%                  model
%
% 	Outputs:
%		blockList  The cell array of reads and writes corresponding to the
%		           input "block"
\end{lstlisting}
% >> End auto gen <<
	\subsection{findDataStoreMemoryRCR}
% >> Start auto gen <<
\begin{lstlisting}
function mem = findDataStoreMemoryRCR(obj, block, flag)
% FINDDATASTOREMEMORY Find the Data Store Memory block of a Data Store
% Read or Write block.
%
% 	Inputs:
% 		obj    The reachcoreach object containing data store mappings
%       block  The data store read or write block of interest
%       flag   The flag indicating whether shadowing data stores are in the
%              model
%
% 	Outputs:
%		mem    The data store memory block corresponding to input "block"
\end{lstlisting}
% >> End auto gen <<
	
	\subsection{GroundAndTerminatePorts}
% >> Start auto gen <<
\begin{lstlisting}
function GroundAndTerminatePorts(sys)
    % GROUNDANDTERMINATEPORTS Ground and terminate all unconnected ports in
    % a system. I.e. For each unconnected input port, create a Ground block
    % and connect that block to the port, and for each unconnected output
    % port, create a Terminator block and connect that block to the port.
    %
    % 	Inputs:
    % 		sys     Simulink system (fullname or handle) for which to
    %               ground and terminate unconnected ports.
    %
    % 	Outputs:
    %		N/A
\end{lstlisting}
% >> End auto gen <<
	
	\subsection{hilite\_system\_notopen}
% >> Start auto gen <<
\begin{lstlisting}
function hilite_system_notopen(sys,hilite,varargin)
%HILITE_SYSTEM_NOTOPEN Highlight a Simulink object.
%   HILITE_SYSTEM_NOTOPEN(SYS) highlights a Simulink object by WITHOUT opening the system
%   window that contains the object and then highlighting the object using the
%   HiliteAncestors property. This is a modification of the original function,
%   described below:
%
%   HILITE_SYSTEM_NOTOPEN(SYS) highlights a Simulink object by first opening the system
%   window that contains the object and then highlighting the object using the
%   HiliteAncestors property.
%
%   You can specify the highlighting options as additional right hand side
%   arguments to HILITE_SYSTEM_NOTOPEN.  Options include:
%
%     default     highlight with the 'default' highlight scheme
%     none        turn off highlighting
%     find        highlight with the 'find' highlight scheme
%     unique      highlight with the 'unique' highlight scheme
%     different   highlight with the 'different' highlight scheme
%     user1       highlight with the 'user1' highlight scheme
%     user2       highlight with the 'user2' highlight scheme
%     user3       highlight with the 'user3' highlight scheme
%     user4       highlight with the 'user4' highlight scheme
%     user5       highlight with the 'user5' highlight scheme
%
%   To alter the colors of a highlighting scheme, use the following command:
%
%     set_param(0, 'HiliteAncestorsData', HILITEDATA)
%
%   where HILITEDATA is a MATLAB structure array with the following fields:
%
%     HiliteType       one of the highlighting schemes listed above
%     ForegroundColor  a color string (listed below)
%     BackgroundColor  a color string (listed below)
%
%   Available colors to set are 'black', 'white', 'red', 'green', 'blue',
%   'yellow', 'magenta', 'cyan', 'gray', 'orange', 'lightBlue', and
%   'darkGreen'.
%
%   Examples:
%
%       % highlight the subsystem 'f14/Controller/Stick Prefilter'
%       HILITE_SYSTEM_NOTOPEN('f14/Controller/Stick Prefilter')
%
%       % highlight the subsystem 'f14/Controller/Stick Prefilter'
%       % in the 'error' highlighting scheme.
%       HILITE_SYSTEM_NOTOPEN('f14/Controller/Stick Prefilter', 'error')
%
%   See also OPEN_SYSTEM, FIND_SYSTEM, SET_PARAM
\end{lstlisting}
% >> End auto gen <<
	
	\subsection{reachCoreachGUI}
% >> Start auto gen <<
\begin{lstlisting}
function varargout = reachCoreachGUI(varargin)
% REACHCOREACHGUI MATLAB code for reachCoreachGUI.fig
%      REACHCOREACHGUI, by itself, creates a new REACHCOREACHGUI or raises the existing
%      singleton*.
%
%      H = REACHCOREACHGUI returns the handle to a new REACHCOREACHGUI or the handle to
%      the existing singleton*.
%
%      REACHCOREACHGUI('CALLBACK',hObject,eventData,handles,...) calls the local
%      function named CALLBACK in REACHCOREACHGUI.M with the given input arguments.
%
%      REACHCOREACHGUI('Property','Value',...) creates a new REACHCOREACHGUI or raises the
%      existing singleton*.  Starting from the left, property value pairs are
%      applied to the GUI before reachCoreachGUI_OpeningFcn gets called.  An
%      unrecognized property name or invalid value makes property application
%      stop.  All inputs are passed to reachCoreachGUI_OpeningFcn via varargin.
%
%      *See GUI Options on GUIDE's Tools menu.  Choose "GUI allows only one
%      instance to run (singleton)".
%
% See also: GUIDE, GUIDATA, GUIHANDLES
\end{lstlisting}
% >> End auto gen <<
	
	\subsection{Diff}
		The Diff folder in the source of the tool contains the functions related with performing the Reach/Coreach option on models starting from the differences between those models rather than selecting starting points manually.
		\subsubsection{Coreach\_Diff}
% >> Start auto gen <<
\begin{lstlisting}
function [oldCoreachedObjects, newCoreachedObjects, diffTree] = Coreach_Diff(oldModel, newModel, highlight, direction, diffTree)
    % COREACH_DIFF Identifies blocks and lines in oldModel and newModel that
    % potentially impact the components changed between the models.
    % 
    % Inputs:
    %   oldModel    The original version of a model.
    %   newModel    The new version of a model.
    %   highlight   [Optional] Indicates whether or not to highlight the
    %               differences and impacts. Default: 1 to highlight differences
    %               with DarkGreen foreground and Red background and highlight
    %               impacts of those differences with Yellow foreground and Red
    %               background; use 0 for no highlighting.
    %   direction   [Optional] Indicates direction of analysis. Default: 1 for
    %               upstream analysis (Coreach), 0 for downstream analysis
    %               (Reach).
    %   diffTree    [Optional] Result of:
    %                   slxmlcomp.compare(oldModel,newModel)
    %               Only used to speed up results.
    %
    % Outputs:
    %   oldCoreachedObjects Handles of blocks and lines in oldModel that 
    %                       potentially impact the changes.
    %   newCoreachedObjects Handles of blocks and lines in newModel that 
    %                       potentially impact the changes.
    %   diffTree            Tree generated from:
    %                           slxmlcomp.compare(oldModel,newModel)
    %                       Can be passed back in on future calls using the same
    %                       models to speed up results.
    % 
\end{lstlisting}
% >> End auto gen <<
		\subsubsection{get\_diffs\_for\_reachcoreach}
% >> Start auto gen <<
\begin{lstlisting}
function [oldBlocks, oldLines, newBlocks, newLines] = get_diffs_for_reachcoreach(model1, model2, diffTree)
    %
    % Inputs:
    %   model1
    %   model2
    %   diffTree    [Optional] Result of:
    %                   slxmlcomp.compare(oldModel,newModel)
    %               Only used to speed up results.
\end{lstlisting}
% >> End auto gen <<
		\subsubsection{highlight\_model\_diffs}
% >> Start auto gen <<
\begin{lstlisting}
function [oldBlocks, oldLines, newBlocks, newLines, oldSubs, newSubs] = highlight_model_diffs(model1, model2, diffTree) %oldBlocks, oldLines, newBlocks, newLines)
    %
\end{lstlisting}
% >> End auto gen <<
		\subsubsection{model\_diff}
% >> Start auto gen <<
\begin{lstlisting}
function diffStruct = model_diff(oldModel, newModel, diffTree)
    % MODEL_DIFF Performs a diff between 2 models and gets changes to blocks,
    % lines, and ports.
    %
    % Inputs:
    %   oldModel    Simulink model.
    %   newModel    Simulink model to treat as an updated version of
    %               oldModel.
    %   diffTree    [Optional] Result of:
    %                   slxmlcomp.compare(oldModel,newModel)
    %               Only used to speed up results.
    %
    % Outputs:
    % 	diffStruct  Struct containing all blocks, lines, and ports that have
    %               changed between oldModel and newModel. The fields are
    %               explained below.
    %       diffStruct.comparisonRoot - xmlcomp.Edits object representing
    %               the model changes using MATLAB's built-in structure.
    %       diffStruct.blocks - Struct used to list the changed blocks.
    %           diffStruct.blocks.add - Struct used to list the added blocks.
    %                   diffStruct.blocks.add.new - Cell array of blocks added
    %                           to the new model.
    %                   diffStruct.blocks.add.old - Cell array of blocks added
    %                           to the old model (note this should always be
    %                           empty, but is here for consistency in the
    %                           diffStruct field structure e.g.
    %                           diffStruct.blocks.del.old is not always empty).
    %                   diffStruct.blocks.add.all - Cell array of blocks added
    %                           to either model.
    %           diffStruct.blocks.del - Struct used to list the deleted blocks;
    %                   has the same field structure as diffStruct.blocks.add.
    %           diffStruct.blocks.mod - Struct used to list the modified blocks;
    %                   has the same field structure as diffStruct.blocks.add.
    %           diffStruct.blocks.mod0 - Struct used to list the modified blocks
    %                   excluding SubSystems just with changed number of ports;
    %                   has the same field structure as diffStruct.blocks.add.
    %           diffStruct.blocks.rename - Struct used to list the renamed
    %                   blocks;
    %                   has the same field structure as diffStruct.blocks.add.
    %       diffStruct.lines - Struct used to list the changed lines;
    %               has the same field structure as diffStruct.blocks.
    %       diffStruct.ports - Struct used to list the changed ports;
    %               has the same field structure as diffStruct.blocks.
    %       diffStruct.notes - Struct used to list the changed annotations;
    %               has the same field structure as diffStruct.blocks.
    %       Note: blocks are ultimately recorded in a cell array, lines and
    %       ports are ultimately recorded in a numeric array.
    %
\end{lstlisting}
% >> End auto gen <<
		\subsubsection{Reach\_Diff}
% >> Start auto gen <<
\begin{lstlisting}
function [oldReachedObjects, newReachedObjects, diffTree] = Reach_Diff(oldModel, newModel, highlight, direction, diffTree)
    % REACH_DIFF Identifies blocks and lines in oldModel and newModel that are 
    % potentially impacted by the changes made between the models.
    % 
    % Inputs:
    %   oldModel    The original version of a model.
    %   newModel    The new version of a model.
    %   highlight   [Optional] Indicates whether or not to highlight the
    %               differences and impacts. Default: 1 to highlight differences
    %               with DarkGreen foreground and Red background and highlight
    %               impacts of those differences with Yellow foreground and Red
    %               background; use 0 for no highlighting.
    %   direction   [Optional] Indicates direction of analysis. Default: 0 for
    %               downstream analysis (Reach), 1 for upstream analysis
    %               (Coreach).
    %   diffTree    [Optional] Result of:
    %                   slxmlcomp.compare(oldModel,newModel)
    %               Only used to speed up results.
    %
    % Outputs:
    %   oldReachedObjects   Handles of blocks and lines in oldModel that are
    %                       potentially impacted.
    %   newReachedObjects   Handles of blocks and lines in newModel that are
    %                       potentially impacted.
    %   diffTree            Tree generated from:
    %                           slxmlcomp.compare(oldModel,newModel)
    %                       Can be passed back in on future calls using the same
    %                       models to speed up results.
    % 
\end{lstlisting}
% >> End auto gen <<
	
	\subsection{ReachUtility}
		The ReachUtility folder in the source of the tool is intended for functions that can be used tangentially to the tool, but that do not represent core functionality.
		\subsubsection{highlight\_blocks}
% >> Start auto gen <<
\begin{lstlisting}
function highlight_blocks(blocks, background, foreground)
\end{lstlisting}
% >> End auto gen <<
		
	\subsection{Utility}
		The Utility folder is for files from our overall tools utilities folder.
		If any functions are saved here, they should not be edited unless at least also editing the version in the general tools utility folder.
		\subsubsection{gcbs}
% >> Start auto gen <<
\begin{lstlisting}
function sels = gcbs
    %sels = gcbs
    %returns a cell array of all currently selected blocks
    %limited to the subsystem established by GCB.
    %C. Hecker/11Dec06
\end{lstlisting}
% >> End auto gen <<
		\subsubsection{gcls}
% >> Start auto gen <<
\begin{lstlisting}
function sels = gcls
    % GCLS Get all currently selected lines.
    %
    %   Inputs:
    %       N/A
    %
    %   Outputs:
    %       sels   Numeric array of line handles.
    %
    %   Example:
    %       >> lines = gcls
    %
    %   lines =
    %       26.0001
    %       28.0004
\end{lstlisting}
% >> End auto gen <<
	
\end{document}